\documentclass[twocolumn, a4paper]{UECIEresume}
\usepackage[dvipdfmx]{graphicx}
\usepackage{graphicx}
\usepackage{amsmath}
\usepackage{txfonts}

\title{卒論,修論予稿フォーマット}
\date{20yy 年 mm 月 dd 日}
\affiliation{I類 XXX プログラム}
%\affiliation{総合情報学科 XXX コース}
%\affiliation{情報学専攻 XXX プログラム}
\supervisor{AAA 教授,BBB 准教授,CCC 助教}
\studentid{000000}
\author{YYY}
\headtitle{20yy 年度 I類 卒研中間発表}
%\headtitle{20yy 年度 総合情報学科 卒研中間発表}
%\headtitle{20yy 年度 I類 卒研発表}
%\headtitle{20yy 年度 総合情報学科 卒研発表}
%\headtitle{20yy 年度 情報学専攻 修士論文中間発表}
%\headtitle{20yy 年度 情報学専攻 修士論文発表}

\begin{document}
\maketitle

\section{はじめに}

「はじめに」の部分はモデルの導入部,背景などを中心に自分の研究の位置づけを書いて下さい.
必要であれば,引用文献を使って自分の立ち位置を記述することが重要です\cite{Kinoshita}.
研究内容をいきなりモデルや実験の説明から始めてはいけません.
研究目的を達成するために,どのようなアプローチをとるのか,
これから述べる方法で,何が何処まで明らかに(あるいは期待)出来るかなどを書く必要があるでしょう.

\section{フォーマット}

「はじめに」の導入の次に実験手法やモデルについての説明を行います.
これには図などを交えて記述することが望ましいでしょう.

また適切な所で段落を切って下さい.段落はロジックのひとまとまりを表すもので,
読者にどう読ませるかを指示するものです.これが不適切だと文書は読みにくくなります.

文書のフォーマットを以下に記します.
\begin{itemize}
  \item サイズは A4.卒論予稿は 1 ないし 2 ページ,修論予稿は 2 ページ
  \item \textbf{2段組を原則とし,おおよそ1行24文字,1ページ45行程度}.文字を詰め込みすぎるものや逆に行間が広すぎるものは望ましくない
  \item マージンは上下 25mm, 左右 20mm
  \item タイトルは,表題,発表者(プログラム名,学籍番号,氏名),指導教員を明記すること
  \item 図や表に関しても上記をはみ出さないようにしてください
  \item 日本語フォントはタイトルがゴシック 12pt, 本文が 9〜10 pt程度とします
  \item ヘッダは表ページ,左側に ``20yy 年度 I類 卒研中間発表'' のような発表会名,右側は学籍番号を入れることとします.裏ページに関しては何も設定しなくても構いません
  \item 各セクションの表題はゴシックを用いて判るようにすること
\end{itemize}
このサンプルファイルに合うような体裁で記述して下さい.
\LaTeX を用いる場合は,添付クラスファイルを用いると楽です.
(日本語の場合は jsarticle.cls が必要です.)

数式を文中に入れる場合は,必ず数式モードを使うようにして下さい.

\section{結果の提示など}

実験結果などは,わかりやすく,図や表を使ってまとめて下さい.予稿は殆どスペースがないので,
これ1枚あれば説明できるという図表を貼り付けると効果的です.
また図表を貼りつけた場合は,キャプションを入れるとともに,
かならず本文中でも説明を行なって下さい.

\section{まとめ}

最後は簡潔に研究成果をまとめて下さい.将来の課題などもあれば書いても良いですが,あまり課題を書きすぎると逆効果になりますのでほどほどにしておきましょう.

また,引用文献はキチンと入れましょう\cite{Kinoshita}.引用は,先人に対するリスペクトなので,よほど独立性が高い研究でない限り必要となります.

{\small
\bibliographystyle{jplain}
\bibliography{template}
}
\end{document}
